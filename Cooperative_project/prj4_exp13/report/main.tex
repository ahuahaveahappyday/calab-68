% 这是中国科学院大学计算机科学与技术专业《计算机组成原理(研讨课)》使用的实验报告 Latex 模板
% 本模板与 2024 年 2 月 Jun-xiong Ji 完成, 更改自由 Shing-Ho Lin 和 Jun-Xiong Ji 于 2022 年 9 月共同完成的基础物理实验模板
% 如有任何问题, 请联系: jijunxoing21@mails.ucas.ac.cn
% This is the LaTeX template for report of Experiment of Computer Organization and Design courses, based on its provided Word template. 
% This template is completed on Febrary 2024, based on the joint collabration of Shing-Ho Lin and Junxiong Ji in September 2022. 
% Adding numerous pictures and equations leads to unsatisfying experience in Word. Therefore LaTeX is better. 
% Feel free to contact me via: jijunxoing21@mails.ucas.ac.cn

\documentclass[11pt]{article}

\usepackage[a4paper]{geometry}
\geometry{left=2.0cm,right=2.0cm,top=2.5cm,bottom=2.5cm}

\usepackage{ctex} % 支持中文的LaTeX宏包
\usepackage{amsmath,amsfonts,graphicx,subfigure,amssymb,bm,amsthm,mathrsfs,mathtools,breqn} % 数学公式和符号的宏包集合
\usepackage{algorithm,algorithmicx} % 算法和伪代码
\usepackage[noend]{algpseudocode} % 算法和伪代码
\usepackage{fancyhdr} % 自定义页眉页脚
\usepackage[framemethod=TikZ]{mdframed} % 创建带边框的框架
\usepackage{fontspec} % 字体设置
\usepackage{adjustbox} % 调整盒子大小
\usepackage{fontsize} % 设置字体大小
\usepackage{tikz,xcolor} % 绘制图形和使用颜色
\usepackage{multicol} % 多栏排版
\usepackage{multirow} % 表格中合并单元格
\usepackage{pdfpages} % 插入PDF文件
\usepackage{listings} % 在文档中插入源代码
\usepackage{wrapfig} % 文字绕排图片
\usepackage{bigstrut,multirow,rotating} % 支持在表格中使用特殊命令
\usepackage{booktabs} % 创建美观的表格
\usepackage{circuitikz} % 绘制电路图
\usepackage{zhnumber} % 中文序号(用于标题)
\usepackage{tabularx} % 表格折行

\definecolor{dkgreen}{rgb}{0,0.6,0}
\definecolor{gray}{rgb}{0.5,0.5,0.5}
\definecolor{mauve}{rgb}{0.58,0,0.82}
\lstset{
  frame=tb,
  aboveskip=3mm,
  belowskip=3mm,
  showstringspaces=false,
  columns=flexible,
  framerule=1pt,
  rulecolor=\color{gray!35},
  backgroundcolor=\color{gray!5},
  basicstyle={\small\ttfamily},
  numbers=none,
  numberstyle=\tiny\color{gray},
  keywordstyle=\color{blue},
  commentstyle=\color{dkgreen},
  stringstyle=\color{mauve},
  breaklines=true,
  breakatwhitespace=true,
  tabsize=3,
}

% 轻松引用, 可以用\cref{}指令直接引用, 自动加前缀. 
% 例: 图片label为fig:1
% \cref{fig:1} => Figure.1
% \ref{fig:1}  => 1
\usepackage[capitalize]{cleveref}
% \crefname{section}{Sec.}{Secs.}
\Crefname{section}{Section}{Sections}
\Crefname{table}{Table}{Tables}
\crefname{table}{Table.}{Tabs.}

% \setmainfont{Palatino Linotype.ttf}
% \setCJKmainfont{SimHei.ttf}
% \setCJKsansfont{Songti.ttf}
% \setCJKmonofont{SimSun.ttf}
\punctstyle{kaiming}
% 偏好的几个字体, 可以根据需要自行加入字体ttf文件并调用

\renewcommand{\emph}[1]{\begin{kaishu}#1\end{kaishu}}

% 对 section 等环境的序号使用中文
\renewcommand \thesection{\zhnum{section}、}
\renewcommand \thesubsection{\arabic{section}}


%%%%%%%%%%%%%%%%%%%%%%%%%%%
%改这里可以修改实验报告表头的信息
\newcommand{\name}{艾华春, 李霄宇, 王敬华}
\newcommand{\studentNum}{2022K8009916011,2022K8009929029,2022K8009925009}
\newcommand{\major}{计算机科学与技术}
\newcommand{\labNum}{4}
\newcommand{\labName}{异常和中断的支持}
%%%%%%%%%%%%%%%%%%%%%%%%%%%

\begin{document}

\begin{center}
  \LARGE \bf 中国科学院大学 \\《计算机体系结构基础(研讨课)》实验报告
\end{center}

\begin{center}
  \emph{姓名} \underline{\makebox[10em][c]{\name}} \\
  % 如果名字比较长, 可以修改box的长度"8em"为其他值
  \emph{学号} \underline{\makebox[30em][c]{\studentNum}}\\
  % \emph{专业} \underline{\makebox[15em][c]{\major}}\\
  \emph{实验项目编号} \underline{\makebox[3em][c]{\labNum}}
  \emph{实验名称} \underline{\makebox[30em][c]{\labName}}\\
\end{center}

% \begin{center}
%   \begin{tabularx}{\textwidth}{|lX|}
%     \hline
%     注1: & 撰写此 Word 格式实验报告后以 PDF 格式保存 SERVE CloudIDE 的 \texttt{/home/serve-ide/ cod-lab/reports} 目录下(注意:reports 全部小写)。文件命名规则:\texttt{prjN.pdf},其中 \texttt{prj} 和后缀名 \texttt{pdf} 为小写,\texttt{N} 为1至4的阿拉伯数字。例如:\texttt{prj1.pdf}。PDF 文件大小应控制在 5MB 以内。此外,实验项目5包含多个选做内容,每个选做实验应提交各自的实验报告文件,文件命名规则:\texttt{prj5-projectname.pdf},其中``-''为英文标点符号的短横线。文件命名举例:\texttt{prj5-dma.pdf}。具体要求详见实验项目5讲义。 \\

%     注2: & 使用\texttt{git add}及\texttt{git commit}命令将实验报告\texttt{PDF}文件添加到本地仓库master分支,并通过\texttt{git push}推送到实验课SERVE GitLab远程仓库master分支(具体命令详见实验报告)。 \\

%     注3: & 实验报告模板下列条目仅供参考,可包含但不限定如下内容。实验报告中无需重复描述讲义中的实验流程。\\
%     \hline
%   \end{tabularx}
% \end{center}

  

\section{逻辑电路结构与仿真波形的截图及说明}
\noindent
$\bullet$
\textbf{添加系统调用异常支持、指令和中断处理}。
\begin{enumerate}
  \item 添加控制寄存器CRMD, PRMD, ESTAT, ERA, EENTRY, SAVE0-3
  
  新建一个csrfile的模块,里面包括所有控制寄存器的实现,并在\verb|mycpu_top|中进行例化,与各个流水级处于并列地位。

  模块接口包括两类,用于指令访存的接口和与处理器硬件电路逻辑直接交互的接口。

  \begin{table}[!ht]
    \centering
    \begin{tabular}{|l|l|l|l|l|}
    \hline
        接口名称 & 接口类型 & 含义 & 位宽 & 输入或输出 \\ \hline
        csr\_re & 指令访问 & 读使能 & 1 & input \\ \hline
        csr\_num & 指令访问 & 寄存器号 & 14 & input \\ \hline
        csr\_rvalue & 指令访问 & 寄存器返回值 & 32 & output \\ \hline
        csr\_we & 指令访问 & 写使能 & 1 & input \\ \hline
        csr\_wmask & 指令访问 & 写掩码 & 32 & input \\ \hline
        csr\_wvalue & 指令访问 & 写数据 & 32 & input \\ \hline
        wb\_ex & 硬件电路交互 & 异常处理触发信号 & 1 & input \\ \hline
        wb\_ecode & 硬件电路交互 & 异常类型 & 6 & input \\ \hline
        wb\_esubcode & 硬件电路交互 & 异常类型 & 9 & input \\ \hline
        wb\_pc & 硬件电路交互 & 触发异常的pc值 & 32 & input \\ \hline
        wb\_vaddr & 硬件电路交互 & 触发异常的虚地址 & 32 & input \\ \hline
        ertn\_flush & 硬件电路交互 & ertn指令 & 1 & input \\ \hline
        .... & ~ & ~ & ~ & ~ \\ \hline
    \end{tabular}
\end{table}
在写回级流水级模块wb\_reg中,创建上述所有的接口。
在cpu\_top模块中,将两个模块wb\_reg和csrfile中对应的接口相连接。

最后,在csrfile模块中创建了所有输入输出接口后,按照讲义上的内容,以每个CSR的各个域作为基本单位,
依次实现各个控制寄存器初始化,被指令访问和修改,被硬件电路逻辑访问和修改。

\item 增加控制寄存器ECFG, BADV, TID, TCFG, TVAL, TICLR

补全ECFG和BADV寄存器。
\begin{lstlisting}[language=verilog]
/*---------------------------ECFG---------------------------------------------------*/
always @(posedge clk)begin
    if(~resetn)
        csr_ecfg_lie <= 13'b0;
    else if (csr_we && csr_num == `CSR_ECFG)            // csr_ecfg_lie[10] == 0 
        csr_ecfg_lie <=     csr_wmask[`CSR_ECFG_LIE] & 13'h1bff & csr_wvalue[`CSR_ECFG_LIE]
                            | ~csr_wmask[`CSR_ECFG_LIE] & 13'h1bff & csr_ecfg_lie;
end
/*---------------------------ERA---------------------------------------------------*/
always @(posedge clk)begin
    if(wb_ex)
        csr_era_pc <=       wb_pc;
    else if(csr_we && csr_num == `CSR_ERA)
        csr_era_pc <=       csr_wmask[`CSR_ERA_PC]  & csr_wvalue[`CSR_ERA_PC]
                            | ~csr_wmask[`CSR_ERA_PC]  & csr_era_pc;
end
\end{lstlisting}
假设定时器位数为32位,从全f开始递减,补全TID, TCFG, TVAL, TICLR寄存器。
\begin{lstlisting}[language=verilog]
/*---------------------------TID-------------------------------------------------------*/           //add TID
always @(posedge clk)begin
    if(~resetn)
        csr_tid_tid <= 32'b0;
    else if(csr_we && csr_num == `CSR_TID)
        csr_tid_tid <= csr_wmask[`CSR_TID_TID] & csr_wvalue[`CSR_TID_TID]
                        | ~csr_wmask[`CSR_TID_TID] & csr_tid_tid;
end

/*---------------------------TCFG------------------------------------------------------*/           //add TCFG
always @(posedge clk)begin
    if(~resetn)
        csr_tcfg_en <= 1'b0;
    else if(csr_we && csr_num==`CSR_TCFG)
        csr_tcfg_en <= csr_wmask[`CSR_TCFG_EN] & csr_wvalue[`CSR_TCFG_EN]
                        | ~csr_wmask[`CSR_TCFG_EN] & csr_tcfg_en;
    
    if(csr_we && csr_num==`CSR_TCFG)begin
        csr_tcfg_periodic <= csr_wmask[`CSR_TCFG_PERIOD] & csr_wvalue[`CSR_TCFG_PERIOD]
                            | ~csr_wmask[`CSR_TCFG_PERIOD] & csr_tcfg_periodic;
        csr_tcfg_initval  <= csr_wmask[`CSR_TCFG_INITV] & csr_wvalue[`CSR_TCFG_INITV]
                            | ~csr_wmask[`CSR_TCFG_INITV] & csr_tcfg_initval;
    end
end

/*---------------------------TVAL------------------------------------------------------*/           //add TVAL
assign tcfg_next_value = csr_wmask[31:0] & csr_wvalue[31:0]
                        | ~csr_wmask[31:0] & {csr_tcfg_initval,csr_tcfg_periodic,csr_tcfg_en};      //value of TCFG in the next clk

always @(posedge clk)begin
    if(~resetn)
        timer_cnt <= 32'hffffffff;
    else if(csr_we && csr_num==`CSR_TCFG && tcfg_next_value[`CSR_TCFG_EN])
        timer_cnt <= {tcfg_next_value[`CSR_TCFG_INITV],2'b0};
    else if(csr_tcfg_en && timer_cnt!=32'hffffffff) begin
        if(timer_cnt[31:0]==32'b0 && csr_tcfg_periodic)
            timer_cnt <= {csr_tcfg_initval,2'b0};
        else
            timer_cnt <= timer_cnt -1'b1;
    end
end

assign csr_tval = timer_cnt[31:0];

/*---------------------------TICLR------------------------------------------------------*/           //add TICLR
assign csr_ticlr_clr = 1'b0;
\end{lstlisting}

\item 增加csrrd,csrwr,csrxchg指令

在ID流水级模块中,通过译码操作,解析出\verb|csr_we, csr_re, csr_num, csr_wmask|信息,并创建从ID流水级到WB流水级的
逐级传递的数据通路,用来传递这些
信号。
\begin{lstlisting}[language=verilog]
  assign id_csr_re  = inst_csrrd || inst_csrwr || inst_csxchg || inst_ertn;   // 控制寄存器读使能
  assign id_csr_num = inst_ertn ?     14'h6           // CSR_ERA
                        :id_excep_en ?   14'hc           // CSR_EENTRY
                        :id_read_TID ?   14'h40          // CSR_TID
                                        :csr_num;
  assign id_csr_we  = inst_csrwr || inst_csxchg;  // 控制寄存器写使能
  assign id_csr_wmask = inst_csxchg ? rj_value: ~32'b0; // 控制寄存器写掩码

\end{lstlisting}

在原来的通用寄存器读出数据rj\_valuw, rkd\_value的逐流水级传递数据通路的基础上,延长其至写回级,作为控制寄存器访问的数据和掩码。

例如,原来rkd\_value的逐级传递到wb流水级为止,作为写回mem的数据。将其继续逐级传递到wb流水级,
并且作为csr控制寄存器的写数据。
\begin{lstlisting}[language=verilog]
  assign csr_wvalue = rkd_value;      // rd 寄存器的旧值作为控制寄存器的写数据。
\end{lstlisting}



\item 添加ertn和syscall指令

在id级添加相应的译码逻辑,生成将当前指令是ertn指令的信号,以及异常使能信号,异常类型相关信号。并依附于流水级逐级传递至写回级模块,通过第一步定义的接口,与csrfile模块
中的对应接口相连接。
\begin{lstlisting}[language=verilog]
  assign id_ertn_flush = inst_ertn;   // 当前指令是ertn

  assign id_excep_INT     =   has_int;        // 记录中断信号
  assign id_excep_SYSCALL =   inst_syscall;   // 记录该条指令是否存在SYSCALL异常
  assign id_excep_BRK     =   inst_break;     // 记录该条指令是否存在BRK异常
  assign id_excep_INE     =   no_inst;        // 记录该条指令是否存在INE异常
  assign id_excep_en =        id_excep_INT | id_excep_SYSCALL | id_excep_BRK | id_excep_INE | if_excep_en;         //只要有一个异常就置1
\end{lstlisting}

并向每个流水级增加从wb发出的清空流水级的信号flush。

从wb流水级向id流水级传递异常跳转地址era,如图所示
\begin{figure}[h!]
  \centering
  \includegraphics[width=0.9\textwidth]{./fig/fig1.png}
  \caption{syscall和ertn增加的部分数据通路}
\end{figure}
每个流水级,接受到清空流水线的信号时,将当前流水级的valid置0。

在IF流水级,如果接受到清空流水线的信号,将下一个pc值设置为从wb传来的返回的异常处理地址。
\begin{lstlisting}[language=verilog]
  assign pre_pc           =   flush ? wb_csr_rvalue // 将下一个pc值设置为从wb传来的返回的异常处理地址
                            : br_taken ? br_target 
                            : seq_pc;
\end{lstlisting}

\item  添加取指地址错(ADEF)、地址非对齐(ALE)、断点(BRK)和指令不存在(INE)异常的支持

从每个例外对应的判断阶段开始,其和其之后的每个阶段都添加一个1位宽的用于记录该条指令是否发生该种例外的控制信号。\par
ADEF在pre-IF级进行判断,当取值地址不为4字节对齐时,产生ADEF例外,并在WB阶段把错误地址传给BADV寄存器。
\begin{lstlisting}[language=verilog]
    assign pre_if_excep_ADEF   =     pre_pc[0] | pre_pc[1];   // 记录该条指令是否存在ADEF异常
    assign pre_if_excep_en =        pre_if_excep_ADEF;
\end{lstlisting}
INE和BRK均在ID阶段判断,当发现是相应指令时产生例外。
\begin{lstlisting}[language=verilog]
// 中断,系统调用,断点,指令不存在异常处理
    assign id_excep_INT     =   has_int;        // 记录中断信号
    assign id_excep_SYSCALL =   inst_syscall;   // 记录该条指令是否存在SYSCALL异常
    assign id_excep_BRK     =   inst_break;     // 记录该条指令是否存在BRK异常
    assign id_excep_INE     =   no_inst;        // 记录该条指令是否存在INE异常
    assign id_excep_en =        id_excep_INT | id_excep_SYSCALL | id_excep_BRK | id_excep_INE | if_excep_en;         //只要有一个异常就置1
    assign id_excep_esubcode =  9'h0;
\end{lstlisting}
ALE在EX阶段判断,当取半字时地址最低位为1,或取字时地址最后两位不全为0,则产生ALE例外,并记录错误地址,等到WB阶段将其传给BADV寄存器。
\begin{lstlisting}[language=verilog]
// 地址非对齐异常处理
    assign ex_excep_ALE = (ex_op_st_ld_h & ex_alu_result[0]) | (ex_op_st_ld_w & (ex_alu_result[1] | ex_alu_result[0]));     // 记录该条指令是否存在ALE异常
    assign ex_excep_en = ex_excep_ALE | id_excep_en;
    
    assign ex_vaddr = {32{ex_read_counter && ~ex_read_counter_low}} & counter[63:32] | 
                      {32{ex_read_counter && ex_read_counter_low}}  & counter[31: 0] |
                      {32{~ex_read_counter}} & ex_alu_result;
\end{lstlisting}

\item 计数器的添加

添加counter.v文件,存放计数器Stable\_Counter,没经过一个时钟周期自增1。
\begin{lstlisting}[language=verilog]
always @(posedge clk)begin
    if(~resetn)
        time_counter <= 64'b0;
    else
        time_counter <= time_counter + 1'b1;
end
\end{lstlisting}

\item 添加rdcntvl.w、rdcntvh.w和rdcntid指令

添加控制信号id\_read\_counter, id\_read\_counter\_low, id\_read\_TID,分别用于记录指令是否需要读取计数器的值,是否要读取计数器的低32位,指令是否要读取计数器ID。
\begin{lstlisting}[language=verilog]
    assign id_read_counter     = inst_rdcntvl_w | inst_rdcntvh_w;
    assign id_read_counter_low = inst_rdcntvl_w;
    assign id_read_TID         = inst_rdcntid; 
\end{lstlisting}
计数器的值在EX阶段根据ex\_read\_counter\_low完成读入。
\begin{lstlisting}[language=verilog]
// 读计数器
    assign ex_counter_result = ex_read_counter_low ? counter[31:0] : counter[63:32];            //处理rdcntvl.w rdcntvh.w指令
\end{lstlisting}
TID的值在WB阶段完成读入。
\begin{lstlisting}[language=verilog]
    assign final_rf_wdata = wb_csr_re   ? csr_rvalue : 
                            wb_read_TID ? csr_rvalue : wb_rf_wdata;             //add csr_tid_rvalue for rdcntid.w
\end{lstlisting}

\item 添加中断处理

 实际上,中断处理可以看作一种特殊的异常,其主要的任务也与异常的处理一致
 首先在控制寄存器(csrfile.v)中设置中断处理信号:
 \begin{lstlisting}[language=verilog]
    assign has_int = ((csr_estat_is[12:0] & csr_ecfg_lie[12:0]) != 13'b0)&& (csr_crmd_ie == 1'b1);
 \end{lstlisting}
 硬件中断通过设备或中断控制器将高电平有效的中断信号连接到 8 个输入引脚上,
ESTAT 控制状态寄存器 IS的 9..2 这八位直接对中断输入引脚的信号采样。软件中断通过 CSR 写指令对ESTAT 状态控制寄存器 IS 域的 1..0 
这两位写 1 或写 0 进行控制。定时器中断的状态记录在 ESTAT 控制状态寄存器 IS 域的第 11 位,其中这些寄存器一一对应着cpu外部的引脚,
\begin{lstlisting}[language=verilog]
    assign hw_int_in = 8'b0;//硬中断接口
    assign ipi_int_in = 1'b0;//定时器中断接口
    always @(posedge clk)begin
    if(~resetn)
        csr_estat_is[1:0] <= 2'b0;
    else if(csr_we && csr_num == `CSR_ESTAT)
        csr_estat_is[1:0] <= csr_wmask[`CSR_ESTAT_IS10]  & csr_wvalue[`CSR_ESTAT_IS10]
                            | ~csr_wmask[`CSR_ESTAT_IS10]  & csr_estat_is[1:0];

    csr_estat_is[9:2]   <= hw_int_in[7:0];          // come from hardware sampling
    csr_estat_is[10]    <= 1'b0;                    // reserved

    if(timer_cnt[31:0] == 32'b0)                     // time counter interrupt
        csr_estat_is[11] <= 1'b1;
    else if(csr_we && csr_num == `CSR_TICLR && csr_wmask[`CSR_TICLR_CLR] && csr_wvalue[`CSR_TICLR_CLR])
        csr_estat_is[11] <= 1'b0;

    csr_estat_is[12]    <= ipi_int_in;   

end
\end{lstlisting}
然后中断处理信号在ID阶段生成,并且和其他异常一起组成excep_en信号流向后面的流水级
\begin{lstlisting}[language=verilog]
    // 中断,系统调用,断点,指令不存在异常处理
    assign id_excep_INT     =   has_int;        // 记录中断信号
    assign id_excep_SYSCALL =   inst_syscall;   // 记录该条指令是否存在SYSCALL异常
    assign id_excep_BRK     =   inst_break;     // 记录该条指令是否存在BRK异常
    assign id_excep_INE     =   no_inst;        // 记录该条指令是否存在INE异常
    assign id_excep_en =        id_excep_INT | id_excep_SYSCALL | id_excep_BRK | id_excep_INE | if_excep_en;         //只要有一个异常就置1
    assign id_excep_esubcode =  9'h0;
\end{lstlisting}
最后中断与其他所有的异常处理在WB阶段生成相应的ecode和wb_ex信号(控制流水线清空)
\begin{lstlisting}[language=verilog]
    assign wb_ecode =   wb_excep_INT     ? 6'h0 :       //INT 中断
    wb_excep_ADEF    ? 6'h8 :       //ADEF
    wb_excep_SYSCALL ? 6'hb :       //SYSCALL
    wb_excep_BRK     ? 6'hc :       //BRK
    wb_excep_INE     ? 6'hd :       //INE
    6'h9;                           //ALE                           
    assign wb_esubcode= wb_excep_esubcode;
    assign wb_ex =      wb_excep_en & wb_valid;

\end{lstlisting}
wb_ex信号和ertn指令信号(ertn_flush)从WB流水级传到cpu顶部,指导各个流水级进行清空(flush)操作,也就是将
相应流水及valid拉低
\begin{lstlisting}[language=verilog]
    .flush(ertn_flush || wb_ex),//IF为例
\end{lstlisting}


\end{enumerate}


\vspace{1ex}

\section{实验过程中遇到的问题、对问题的思考过程及解决方法(比如RTL代码中出现的逻辑bug,逻辑仿真和FPGA调试过程中的难点等)}

\noindent
$\bullet$
\textbf{wb流水级的flush有效信号只持续一个clk}。

在清空流水线的设计时,将wb流水级在发出flush信号时的下一个clk时,也要将wb\_valid置为0,避免重复清空流水线。

使得如果IF流水级的allowin由于读后写数据相关等原因为0时,不能及时把\verb|pre_pc = wb_csr_rvalue| 发送给\verb|inst_sram|,而随着下一个clk的wb\_valid拉低,
使从wb传到if的wb\_csr\_rvalue信号也失效。

在这种情况下,不能正确地进行跳转到异常处理地址。

在分析波形图后,确定上述bug后,为if\_allowin添加上规则,使得接收到清空流水线时,立马将allowin拉高,在当前clk发送出pc

\begin{lstlisting}[language=verilog]
  assign if_allowin       =   ~if_valid               // valid是reg类型,接受flush后最快下一个clk才能拉低
                              | if_ready_go & id_allowin    // id_allowin可能由于读后写阻塞,拉低
                              | flush;    // 添加上规则,立马将allowin拉高,在当前clk发送出pc
\end{lstlisting}

\noindent
$\bullet$
\textbf{指令译码定义太宽泛导致异常判断出错}。

问题报错如下:
\begin{figure}[h]
    \centering
    \includegraphics[width=17cm]{fig/1.png}
  \end{figure}
\begin{figure}[H]
    \centering
    \includegraphics[width=15cm]{fig/2.png}
  \end{figure}
查看信号波形如上,按照错误信号定位,可以追溯到是slli.w指令(1c013e38:	0040818a 	slli.w	$r10,$r12,0x0
),这条指令本不应该出现异常
,但是错误的发生了异常信号ex_excep_en进一步追溯发现是由ex_op_st_ld_h信号导致的。
\begin{figure}[H]
    \centering
    \includegraphics[width=15cm]{fig/5.png}
  \end{figure}

查看原来的代码发现,原来的信号定义的过于宽泛,导致出现了很多错误的拉高
\begin{figure}[H]
    \centering
    \includegraphics[width=15cm]{fig/3.png}
  \end{figure}
进行如下修改之后,通过该测试点。
\begin{figure}[H]
    \centering
    \includegraphics[width=15cm]{fig/4.png}
  \end{figure}

  \noindent
  $\bullet$
  \textbf{控制寄存器写入逻辑出错}。

  报错信息如下:
  \begin{figure}[H]
    \centering
    \includegraphics[width=15cm]{fig/6.png}
  \end{figure}
  \begin{figure}[H]
    \centering
    \includegraphics[width=15cm]{fig/7.png}
  \end{figure}
  查看信号波形如上,按照错误信号定位,可以定位到指令(1c0081ac:	0400040c 	csrrd	\$r12,0x1)
  出错,错误的原因是控制寄存器prmd存储的的值出错,向前寻找上一次写改变该值的指令(图中黄线),发现
  在这个周期虽然没有写使能信号,但是寄存器的值发生了神秘的变化,仔细观察代码逻辑,发现原来的代码
  忘记加begin和end导致寄存器赋值逻辑出错,修改后通过。
  \begin{figure}[H]
    \centering
    \includegraphics[width=15cm]{fig/8.png}
  \end{figure}

  \noindent
    $\bullet$
    \textbf{控制寄存器读出逻辑出错}。

    报错信息如下:
    \begin{figure}[H]
        \centering
        \includegraphics[width=15cm]{fig/9.png}
      \end{figure}
      \begin{figure}[H]
        \centering
        \includegraphics[width=15cm]{fig/10.png}
      \end{figure}
    查看信号波形如上,按照错误信号定位,可以定位到指令(1c0722f8:	0400102c 	csrwr	\$r12,0x4)
    出错,错误的原因是读控制寄存的读出值csr_rvalue出错,但是ecfg寄存器内部寄存的值是正确的,仔细观察赋值逻辑,发现一处笔误
    错误的将csr_ecfg_rvalue写作csr_tcfg_rvalue,修改之后通过。
    \begin{figure}[H]
        \centering
        \includegraphics[width=15cm]{fig/11.png}
      \end{figure}

      \noindent
      $\bullet$
      \textbf{trace比对提交的条件出错}。
  
      报错信息如下:
      \begin{figure}[H]
          \centering
          \includegraphics[width=15cm]{fig/12.png}
        \end{figure}
      按照错误信号定位,可以定位到指令(1c074d94:	ffffffff 	0xffffffff)
      出错,错误的原因是:这是一条错误的指令,cpu测出这是一种异常,但是trace比对写回值结果的时候
      是以debug_wb_rf_we为使能信号的,在异常产生的时候,应该优先处理异常,拉低这个信号,因此在这个信号
      内部\&~wb_excep_en信号表示没有出现异常的时候才进行比对,并且对wb_rf_we的前身gr_we进行了补充
      \begin{figure}[H]
          \centering
          \includegraphics[width=15cm]{fig/13.png}
        \end{figure}
        \begin{figure}[H]
            \centering
            \includegraphics[width=15cm]{fig/14.png}
          \end{figure}
          \begin{figure}[H]
            \centering
            \includegraphics[width=15cm]{fig/15.png}
          \end{figure}

       \noindent
        $\bullet$
        \textbf{在EXE、MEM判断出ERTN后未拉低写使能信号        }。
      
          报错信息如下:
          \begin{figure}[H]
              \centering
              \includegraphics[width=15cm]{fig/16.png}
            \end{figure}
            \begin{figure}[H]
              \centering
              \includegraphics[width=15cm]{fig/17.png}
            \end{figure}
          查看信号波形如上,按照错误信号定位,查看当前指令类型(1c008004:	288001ad 	ld.w	$r13,$r13,0)
          load读数据出错说明错误的原因是上一次访存指令错误的写入数据,猜测前序写内存时有误,查看对应数据最近一次被写入的时机:
          找到图中黄线的时刻,是一条st.w指令后面跟着ertn指令
          \begin{lstlisting}
            1c0751c8:	2980127b 	st.w	$r27,$r19,4(0x4)
            1c0751cc:	06483800 	ertn
          \end{lstlisting}
          这下破案了,原因是ertn指令判断后,没有取消内存写使能的信号,导致错误的数据被写入
          \begin{figure}[H]
            \centering
            \includegraphics[width=15cm]{fig/18.png}
          \end{figure}
          \begin{lstlisting}
            assign data_sram_en     = (ex_res_from_mem || ex_mem_we) && ex_valid && ~mem_excep_en && ~wb_excep_en &&~mem_ertn_flush&& ~wb_ertn_flush && ~ex_excep_ALE;
          \end{lstlisting}
          按照书上的要求进行修改,在内存片选信号上面添加对ertn的检查信号之后,通过该测试点。

          \noindent
          $\bullet$
          \textbf{异常处理写入的PC值入口出错}。
        
          报错信息如下:
          \begin{figure}[H]
            \centering
            \includegraphics[width=15cm]{fig/19.png}
          \end{figure}
          \begin{figure}[H]
            \centering
            \includegraphics[width=15cm]{fig/20.png}
          \end{figure}
          查看信号波形如上,按照PC错误跳转信号定位,可以定位到指令(1c0753a8:	28a1148e 	ld.w	$r14,$r4,-1979(0x845))
          这条指令出现了ALE异常,将flush信号拉高,因此清空流水线并且prePC进行取指令,错误的原因是
          异常处理的pc地址出错。追根溯源,可以发现是错误的不加区分的直接使用了csr的读出值csr_rvalue作为了PC。
          正确的方式应该是根据是ertn还是异常采用区分era或者eenery的值
           \begin{lstlisting}[language=verilog]
            assign excep_entry = wb_ex ? csr_eentry_rvalue     //PC在flush时候取指的地址
                  /*ertn_flush*/:csr_era_pc;
          \end{lstlisting}
          修改之后,通过测试点。

          \noindent
          $\bullet$
          \textbf{控制寄存器写入使能信号出错}。
        
          报错信息如下:
          \begin{figure}[H]
            \centering
            \includegraphics[width=15cm]{fig/21.png}
          \end{figure}
          根据错误出现的位置,可以发现指令为(1c0753ac:	5c02673e 	bne	\$r25,\$r30,612(0x264) 1c075610 <inst_error>
          )指令发生了错误的跳转导致出错,原因是通用寄存器的值被前面的指令错误的修改了,定位到上次这个寄存器被修改的位置:

          \begin{figure}[H]
            \centering
            \includegraphics[width=15cm]{fig/22.png}
          \end{figure}
          \begin{figure}[H]
            \centering
            \includegraphics[width=15cm]{fig/23.png}
          \end{figure}
          查看信号波形如上,这里的指令是产生了flush信号,按理来说应该将通用寄存器的写使能拉低,但是这里
          并没有,这就导致了寄存器值的改变。
           \begin{lstlisting}[language=verilog]
            assign wb_to_id_bus = {wb_rf_we & wb_valid & ~wb_ex & ~ertn_flush, wb_rf_waddr, final_rf_wdata};
          \end{lstlisting}
          在写使能处加上了\& ~ertn_flush即可,通过测试点。

          \noindent
          $\bullet$
          \textbf{控制寄存器信号悬空}。
        
          报错信息如下:
          \begin{figure}[H]
            \centering
            \includegraphics[width=15cm]{fig/24.png}
          \end{figure}

          这里发生错误的指令是(1c0753c4:	04001c0c 	csrrd	\$r12,0x7),错误的原因是
          07控制寄存器CSR_BADV的值出现高阻态导致无法读出。
          \begin{figure}[H]
            \centering
            \includegraphics[width=15cm]{fig/25.png}
          \end{figure}

          \begin{lstlisting}[language=verilog]
        always @(posedge clk)begin
         if(wb_ex && wb_ex_addr_err)
            csr_badv_vaddr <=       (wb_ecode == `ECODE_ADE && wb_esubcode == `ESUBCODE_ADEF) ?
                                wb_pc           //  inst fecth error
                                :wb_vaddr;      // mem access error and so on
            end
          \end{lstlisting}
          查看信号波形如上,通过代码这里可以看出是因为wb_vaddr信号悬空没有赋值(实际上通路也没有打通)导致
          出现高阻态。
           \begin{lstlisting}[language=verilog]
            assign ex_vaddr = {32{ex_read_counter && ~ex_read_counter_low}} & counter[63:32] | 
            {32{ex_read_counter && ex_read_counter_low}}  & counter[31: 0] |
            {32{~ex_read_counter}} & ex_alu_result;
          \end{lstlisting}
         实现信号并且打通通路之后,错误解除。
          
         \noindent
          $\bullet$
          \textbf{中断未复位ALU}。
        
          报错信息如下:
          \begin{figure}[H]
            \centering
            \includegraphics[width=15cm]{fig/26.png}
          \end{figure}
          这是一条除法指令( 1c0754f8:	0020418c 	div.w	\$r12,\$r12,\$r16)
          查看信号波形发现恰好该div指令在wb阶段判断出异常:
          异常发生时alu中乘法和除法器的计数器未被复位,导致后续操作出错。应该令alu的resetn信号
          与上~wb_excep_en和~wb_ertn_flush。
           \begin{lstlisting}[language=verilog]
            //alu的实例化
    alu u_alu(
        .clk            (clk       ),
        .resetn         (resetn && ~wb_excep_en  && ~wb_ertn_flush  ),
        .alu_op         (ex_alu_op    ),
        .alu_src1       (ex_alu_src1  ),
        .alu_src2       (ex_alu_src2  ),
        .alu_result     (ex_alu_result),
        .complete       (alu_complete)
    );
          \end{lstlisting}
          修改之后,测试PASS!

         

      
\vspace{1ex}

\section{小组成员分工合作情况}
王敬华负责exp13的中断处理和整体的debug工作

李霄宇负责exp13的异常处理和计时器指令的实现

艾华春负责exp12:添加系统调用异常支持

实验报告为根据每人负责代码的部分,写相应部分的报告。
\end{document}