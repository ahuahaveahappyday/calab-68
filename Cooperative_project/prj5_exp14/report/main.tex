% 这是中国科学院大学计算机科学与技术专业《计算机组成原理(研讨课)》使用的实验报告 Latex 模板
% 本模板与 2024 年 2 月 Jun-xiong Ji 完成, 更改自由 Shing-Ho Lin 和 Jun-Xiong Ji 于 2022 年 9 月共同完成的基础物理实验模板
% 如有任何问题, 请联系: jijunxoing21@mails.ucas.ac.cn
% This is the LaTeX template for report of Experiment of Computer Organization and Design courses, based on its provided Word template. 
% This template is completed on Febrary 2024, based on the joint collabration of Shing-Ho Lin and Junxiong Ji in September 2022. 
% Adding numerous pictures and equations leads to unsatisfying experience in Word. Therefore LaTeX is better. 
% Feel free to contact me via: jijunxoing21@mails.ucas.ac.cn

\documentclass[11pt]{article}

\usepackage[a4paper]{geometry}
\geometry{left=2.0cm,right=2.0cm,top=2.5cm,bottom=2.5cm}

\usepackage{ctex} % 支持中文的LaTeX宏包
\usepackage{amsmath,amsfonts,graphicx,subfigure,amssymb,bm,amsthm,mathrsfs,mathtools,breqn} % 数学公式和符号的宏包集合
\usepackage{algorithm,algorithmicx} % 算法和伪代码
\usepackage[noend]{algpseudocode} % 算法和伪代码
\usepackage{fancyhdr} % 自定义页眉页脚
\usepackage[framemethod=TikZ]{mdframed} % 创建带边框的框架
\usepackage{fontspec} % 字体设置
\usepackage{adjustbox} % 调整盒子大小
\usepackage{fontsize} % 设置字体大小
\usepackage{tikz,xcolor} % 绘制图形和使用颜色
\usepackage{multicol} % 多栏排版
\usepackage{multirow} % 表格中合并单元格
\usepackage{pdfpages} % 插入PDF文件
\usepackage{listings} % 在文档中插入源代码
\usepackage{wrapfig} % 文字绕排图片
\usepackage{bigstrut,multirow,rotating} % 支持在表格中使用特殊命令
\usepackage{booktabs} % 创建美观的表格
\usepackage{circuitikz} % 绘制电路图
\usepackage{zhnumber} % 中文序号(用于标题)
\usepackage{tabularx} % 表格折行

\definecolor{dkgreen}{rgb}{0,0.6,0}
\definecolor{gray}{rgb}{0.5,0.5,0.5}
\definecolor{mauve}{rgb}{0.58,0,0.82}
\lstset{
  frame=tb,
  aboveskip=3mm,
  belowskip=3mm,
  showstringspaces=false,
  columns=flexible,
  framerule=1pt,
  rulecolor=\color{gray!35},
  backgroundcolor=\color{gray!5},
  basicstyle={\small\ttfamily},
  numbers=none,
  numberstyle=\tiny\color{gray},
  keywordstyle=\color{blue},
  commentstyle=\color{dkgreen},
  stringstyle=\color{mauve},
  breaklines=true,
  breakatwhitespace=true,
  tabsize=3,
}

% 轻松引用, 可以用\cref{}指令直接引用, 自动加前缀. 
% 例: 图片label为fig:1
% \cref{fig:1} => Figure.1
% \ref{fig:1}  => 1
\usepackage[capitalize]{cleveref}
% \crefname{section}{Sec.}{Secs.}
\Crefname{section}{Section}{Sections}
\Crefname{table}{Table}{Tables}
\crefname{table}{Table.}{Tabs.}

% \setmainfont{Palatino Linotype.ttf}
% \setCJKmainfont{SimHei.ttf}
% \setCJKsansfont{Songti.ttf}
% \setCJKmonofont{SimSun.ttf}
\punctstyle{kaiming}
% 偏好的几个字体, 可以根据需要自行加入字体ttf文件并调用

\renewcommand{\emph}[1]{\begin{kaishu}#1\end{kaishu}}

% 对 section 等环境的序号使用中文
\renewcommand \thesection{\zhnum{section}、}
\renewcommand \thesubsection{\arabic{section}}


%%%%%%%%%%%%%%%%%%%%%%%%%%%
%改这里可以修改实验报告表头的信息
\newcommand{\name}{艾华春, 李霄宇, 王敬华}
\newcommand{\studentNum}{2022K8009916011,2022K8009929029,2022K8009925009}
\newcommand{\major}{计算机科学与技术}
\newcommand{\labNum}{4}
\newcommand{\labName}{添加类SRAM总线支持}
%%%%%%%%%%%%%%%%%%%%%%%%%%%

\begin{document}

\begin{center}
  \LARGE \bf 中国科学院大学 \\《计算机体系结构基础(研讨课)》实验报告
\end{center}

\begin{center}
  \emph{姓名} \underline{\makebox[10em][c]{\name}} \\
  % 如果名字比较长, 可以修改box的长度"8em"为其他值
  \emph{学号} \underline{\makebox[30em][c]{\studentNum}}\\
  % \emph{专业} \underline{\makebox[15em][c]{\major}}\\
  \emph{实验项目编号} \underline{\makebox[3em][c]{\labNum}}
  \emph{实验名称} \underline{\makebox[30em][c]{\labName}}\\
\end{center}

% \begin{center}
%   \begin{tabularx}{\textwidth}{|lX|}
%     \hline
%     注1: & 撰写此 Word 格式实验报告后以 PDF 格式保存 SERVE CloudIDE 的 \texttt{/home/serve-ide/ cod-lab/reports} 目录下(注意:reports 全部小写)。文件命名规则:\texttt{prjN.pdf},其中 \texttt{prj} 和后缀名 \texttt{pdf} 为小写,\texttt{N} 为1至4的阿拉伯数字。例如:\texttt{prj1.pdf}。PDF 文件大小应控制在 5MB 以内。此外,实验项目5包含多个选做内容,每个选做实验应提交各自的实验报告文件,文件命名规则:\texttt{prj5-projectname.pdf},其中``-''为英文标点符号的短横线。文件命名举例:\texttt{prj5-dma.pdf}。具体要求详见实验项目5讲义。 \\

%     注2: & 使用\texttt{git add}及\texttt{git commit}命令将实验报告\texttt{PDF}文件添加到本地仓库master分支,并通过\texttt{git push}推送到实验课SERVE GitLab远程仓库master分支(具体命令详见实验报告)。 \\

%     注3: & 实验报告模板下列条目仅供参考,可包含但不限定如下内容。实验报告中无需重复描述讲义中的实验流程。\\
%     \hline
%   \end{tabularx}
% \end{center}

  

\section{逻辑电路结构与仿真波形的截图及说明}
\noindent
$\bullet$
\textbf{}。



\vspace{1ex}

\section{实验过程中遇到的问题、对问题的思考过程及解决方法(比如RTL代码中出现的逻辑bug,逻辑仿真和FPGA调试过程中的难点等)}

\noindent
$\bullet$
\textbf{}。


\vspace{1ex}

\section{小组成员分工合作情况}


实验报告为根据每人负责代码的部分,写相应部分的报告。
\end{document}