% 这是中国科学院大学计算机科学与技术专业《计算机组成原理(研讨课)》使用的实验报告 Latex 模板
% 本模板与 2024 年 2 月 Jun-xiong Ji 完成, 更改自由 Shing-Ho Lin 和 Jun-Xiong Ji 于 2022 年 9 月共同完成的基础物理实验模板
% 如有任何问题, 请联系: jijunxoing21@mails.ucas.ac.cn
% This is the LaTeX template for report of Experiment of Computer Organization and Design courses, based on its provided Word template. 
% This template is completed on Febrary 2024, based on the joint collabration of Shing-Ho Lin and Junxiong Ji in September 2022. 
% Adding numerous pictures and equations leads to unsatisfying experience in Word. Therefore LaTeX is better. 
% Feel free to contact me via: jijunxoing21@mails.ucas.ac.cn

\documentclass[11pt]{article}

\usepackage[a4paper]{geometry}
\geometry{left=2.0cm,right=2.0cm,top=2.5cm,bottom=2.5cm}

\usepackage{ctex} % 支持中文的LaTeX宏包
\usepackage{amsmath,amsfonts,graphicx,subfigure,amssymb,bm,amsthm,mathrsfs,mathtools,breqn} % 数学公式和符号的宏包集合
\usepackage{algorithm,algorithmicx} % 算法和伪代码
\usepackage[noend]{algpseudocode} % 算法和伪代码
\usepackage{fancyhdr} % 自定义页眉页脚
\usepackage[framemethod=TikZ]{mdframed} % 创建带边框的框架
\usepackage{fontspec} % 字体设置
\usepackage{adjustbox} % 调整盒子大小
\usepackage{fontsize} % 设置字体大小
\usepackage{tikz,xcolor} % 绘制图形和使用颜色
\usepackage{multicol} % 多栏排版
\usepackage{multirow} % 表格中合并单元格
\usepackage{pdfpages} % 插入PDF文件
\usepackage{listings} % 在文档中插入源代码
\usepackage{wrapfig} % 文字绕排图片
\usepackage{bigstrut,multirow,rotating} % 支持在表格中使用特殊命令
\usepackage{booktabs} % 创建美观的表格
\usepackage{circuitikz} % 绘制电路图
\usepackage{zhnumber} % 中文序号(用于标题)
\usepackage{tabularx} % 表格折行

\definecolor{dkgreen}{rgb}{0,0.6,0}
\definecolor{gray}{rgb}{0.5,0.5,0.5}
\definecolor{mauve}{rgb}{0.58,0,0.82}
\lstset{
  frame=tb,
  aboveskip=3mm,
  belowskip=3mm,
  showstringspaces=false,
  columns=flexible,
  framerule=1pt,
  rulecolor=\color{gray!35},
  backgroundcolor=\color{gray!5},
  basicstyle={\small\ttfamily},
  numbers=none,
  numberstyle=\tiny\color{gray},
  keywordstyle=\color{blue},
  commentstyle=\color{dkgreen},
  stringstyle=\color{mauve},
  breaklines=true,
  breakatwhitespace=true,
  tabsize=3,
}

% 轻松引用, 可以用\cref{}指令直接引用, 自动加前缀. 
% 例: 图片label为fig:1
% \cref{fig:1} => Figure.1
% \ref{fig:1}  => 1
\usepackage[capitalize]{cleveref}
% \crefname{section}{Sec.}{Secs.}
\Crefname{section}{Section}{Sections}
\Crefname{table}{Table}{Tables}
\crefname{table}{Table.}{Tabs.}

% \setmainfont{Palatino Linotype.ttf}
% \setCJKmainfont{SimHei.ttf}
% \setCJKsansfont{Songti.ttf}
% \setCJKmonofont{SimSun.ttf}
\punctstyle{kaiming}
% 偏好的几个字体, 可以根据需要自行加入字体ttf文件并调用

\renewcommand{\emph}[1]{\begin{kaishu}#1\end{kaishu}}

% 对 section 等环境的序号使用中文
\renewcommand \thesection{\zhnum{section}、}
\renewcommand \thesubsection{\arabic{section}}


%%%%%%%%%%%%%%%%%%%%%%%%%%%
%改这里可以修改实验报告表头的信息
\newcommand{\name}{艾华春, 李霄宇, 王敬华}
\newcommand{\studentNum}{2022K8009916011,2022K8009929029,2022K8009925009}
\newcommand{\major}{计算机科学与技术}
\newcommand{\labNum}{6}
\newcommand{\labName}{存储管理单元设计}
%%%%%%%%%%%%%%%%%%%%%%%%%%%

\begin{document}

\begin{center}
  \LARGE \bf 中国科学院大学 \\《计算机体系结构基础(研讨课)》实验报告
\end{center}

\begin{center}
  \emph{姓名} \underline{\makebox[10em][c]{\name}} \\
  % 如果名字比较长, 可以修改box的长度"8em"为其他值
  \emph{学号} \underline{\makebox[30em][c]{\studentNum}}\\
  % \emph{专业} \underline{\makebox[15em][c]{\major}}\\
  \emph{实验项目编号} \underline{\makebox[3em][c]{\labNum}}
  \emph{实验名称} \underline{\makebox[30em][c]{\labName}}\\
\end{center}

% \begin{center}
%   \begin{tabularx}{\textwidth}{|lX|}
%     \hline
%     注1: & 撰写此 Word 格式实验报告后以 PDF 格式保存 SERVE CloudIDE 的 \texttt{/home/serve-ide/ cod-lab/reports} 目录下(注意:reports 全部小写)。文件命名规则:\texttt{prjN.pdf},其中 \texttt{prj} 和后缀名 \texttt{pdf} 为小写,\texttt{N} 为1至4的阿拉伯数字。例如:\texttt{prj1.pdf}。PDF 文件大小应控制在 5MB 以内。此外,实验项目5包含多个选做内容,每个选做实验应提交各自的实验报告文件,文件命名规则:\texttt{prj5-projectname.pdf},其中``-''为英文标点符号的短横线。文件命名举例:\texttt{prj5-dma.pdf}。具体要求详见实验项目5讲义。 \\

%     注2: & 使用\texttt{git add}及\texttt{git commit}命令将实验报告\texttt{PDF}文件添加到本地仓库master分支,并通过\texttt{git push}推送到实验课SERVE GitLab远程仓库master分支(具体命令详见实验报告)。 \\

%     注3: & 实验报告模板下列条目仅供参考,可包含但不限定如下内容。实验报告中无需重复描述讲义中的实验流程。\\
%     \hline
%   \end{tabularx}
% \end{center}

  

\section{逻辑电路结构与仿真波形的截图及说明}
\noindent
$\bullet$
\textbf{TLB模块设计}。
\begin{enumerate}
  \item TLB 总体工作原理\par
  TLB 采用全相联查找表的组织形式。在TLB模块中保存了TLB页表的内容,共16个页表项。并通过2个查找端口,1个读端口,1个写端口和一个INVTLB指令
  端口与外部交换数据。实现了对TLB页表的4种操作:\par
  查找:输入虚双页号vppn、虚地址12位va_bit12与地址空间标识asid,
  得到对应物理页的内容。\par
  写:输入页表项数index,并将输入的页表项内容写到对应的页表项里。\par
  读:输入页表项数index,并得到对应的页表项中的内容。\par
  无效指令INVTLB:使满足opcode对应条件的页表项无效化。\par
  TLB页表中每一个页表项为一个双页,包含比较部分和物理转换部分,结构如下图:
  \begin{figure}[H]
    \centering
    \includegraphics[width=13cm]{fig/1.png}
  \end{figure}
  本次实验中实现的TLB页表中包含16个页表项,
  能保存16对32个页。页大小有2MB与4KB两种,通过ps4MB信号指示。
  使用15组寄存器来保存页表内容:
  \begin{lstlisting}[language=verilog]
    reg [TLBNUM-1:0] tlb_e;
    reg [TLBNUM-1:0] tlb_ps4MB; //pagesize 1:4MB, 0:4KB

    reg [18:0] tlb_vppn [TLBNUM-1:0];
    reg [ 9:0] tlb_asid [TLBNUM-1:0];
    reg        tlb_g    [TLBNUM-1:0];

    reg [19:0] tlb_ppn0 [TLBNUM-1:0];
    reg [ 1:0] tlb_plv0 [TLBNUM-1:0];
    reg [ 1:0] tlb_mat0 [TLBNUM-1:0];
    reg        tlb_d0   [TLBNUM-1:0];
    reg        tlb_v0   [TLBNUM-1:0];

    reg [19:0] tlb_ppn1 [TLBNUM-1:0];
    reg [ 1:0] tlb_plv1 [TLBNUM-1:0];
    reg [ 1:0] tlb_mat1 [TLBNUM-1:0];
    reg        tlb_d1   [TLBNUM-1:0];
    reg        tlb_v1   [TLBNUM-1:0];
  \end{lstlisting}
  \item TLB 读写设计\par

  TLB模块的读写方式与寄存器堆的读写方式十分类似。读TLB的方式是根据传入的index将对应表项各个域的数据通
  过组合逻辑读出;写TLB的方式是根据传入的index和写使能信
  号,在下一拍将要写的各个域的数据写入对应表项相应位置。
  \begin{lstlisting}[language=verilog]
    //read
assign r_e    = tlb_e    [r_index];
assign r_vppn = tlb_vppn [r_index];
assign r_ps   = tlb_ps4MB[r_index]? 6'b010101 : 6'b001100;
assign r_asid = tlb_asid [r_index];
assign r_g    = tlb_g    [r_index];

assign r_ppn0 = tlb_ppn0 [r_index];
assign r_ppn1 = tlb_ppn1 [r_index];
assign r_plv0 = tlb_plv0 [r_index];
assign r_plv1 = tlb_plv1 [r_index];
assign r_mat0 = tlb_mat0 [r_index];
assign r_mat1 = tlb_mat1 [r_index];
assign r_d0   = tlb_d0   [r_index];
assign r_d1   = tlb_d1   [r_index];
assign r_v0   = tlb_v0   [r_index];
assign r_v1   = tlb_v1   [r_index];

//write
wire   w_ps4MB;
assign w_ps4MB=(w_ps == 6'b010101)?1:0;

always @(posedge clk)
    begin
        if(we)
            begin
                tlb_e[w_index]      <= w_e;
                tlb_vppn[w_index]   <= w_vppn;
                tlb_ps4MB[w_index]  <= w_ps4MB;
                tlb_asid[w_index]   <= w_asid;
                tlb_g[w_index]      <= w_g;

                tlb_ppn0[w_index]   <= w_ppn0;
                tlb_ppn1[w_index]   <= w_ppn1;
                tlb_plv0[w_index]   <= w_plv0;
                tlb_plv1[w_index]   <= w_plv1;
                tlb_mat0[w_index]   <= w_mat0;
                tlb_mat1[w_index]   <= w_mat1;
                tlb_d0[w_index]     <= w_d0;
                tlb_d1[w_index]     <= w_d1;
                tlb_v0[w_index]     <= w_v0;
                tlb_v1[w_index]     <= w_v1;
            end
  \end{lstlisting}
  但在读写过程中需要注意的是,由于龙芯架构32位精简版只支持4KB和
  4MB两种页大小,因此尽管TLB表项中的PS域是6bit,但其只有0和1两
  种取值:0表示页大小为4KB(两个物理页大小均为4KB),1表示页大小
  为4MB(两个物理页大小均为2MB)。而TLB模块输入和输出中的PS端口标记
  了页的真实大小,取值分别是6’d21和6’d12,因此在对页表项进行读写
  操作时需要将两者进行转换。\par
  除此之外,TLB 写操作还需要实现TLB invtlb指令的opcode对应的操作:
  \begin{figure}[H]
    \centering
    \includegraphics[width=13cm]{fig/2.png}
  \end{figure}
  按照讲义上面的提示,可以将各操作的匹配分解成若干 “子匹配” 的逻辑组合,具体来说,可得到 4 个 “子匹配”
  的判断条件:(1)cond1——G 域是否等于 0;(2)cond2——G 域是否等于 1;(3)cond3——s1_asid
  是否等于 ASID 域;(4)cond4——s1_vppn 是否匹配 VPPN 和 PS 域
  \begin{lstlisting}[language=verilog]
    generate
    for (i = 0; i < TLBNUM; i = i + 1) begin
       assign cond1[i] = ~tlb_g[i];
       assign cond2[i] =  tlb_g[i];
       assign cond3[i] = s1_asid == tlb_asid[i];
       assign cond4[i] = (s1_vppn[18:9] == tlb_vppn[i][18:9])&&(tlb_ps4MB[i]||(s1_vppn[8:0] == tlb_vppn[i][8:0]));
    end
endgenerate
  \end{lstlisting}
  那么 invtlb op=0、1 的匹配
  条件就可以表达为 cond1||cond2,op=4 的匹配条件可以表达为 cond1\&\&cond3,op=5 的匹配条
  件可以表达为 cond1\&\&cond3\&\&cond4,op=6 的匹配条件可以表达为 (cond2||cond3)\&\&cond4。
  根据cond1~cond4得到match信号如下:
  \begin{lstlisting}[language=verilog]
    assign inv_match = (invtlb_op == 5'h0 || invtlb_op == 5'h1) ? (cond1 | cond2):
    invtlb_op == 5'h2 ? cond2:
    invtlb_op == 5'h3 ? cond1:
    invtlb_op == 5'h4 ? cond1 & cond3:
    invtlb_op == 5'h5 ? cond1 & cond3 & cond4:
    (cond2 | cond3) & cond4;
  \end{lstlisting}
  在invtlb指令有效时,根据输入的opcode选择inv_match,
  将inv_match对应位为1的页表项无效化(e值清零),为0的页表项不变:
  \begin{lstlisting}[language=verilog]
    else if(invtlb_valid && invtlb_op < 5'h7)
    begin
        tlb_e               <= ~inv_match & tlb_e;
    end
  \end{lstlisting}
  \item TLB 查找设计\par
  TLB模块提供了两个通道用于查找,一个通道对应取指阶段的查找,
  另一个通道对应访存阶段的查找。查找的流程是并行化的,即同时
  对16个页表项的奇偶相邻页表的物理转换信息进行比较和查找,并用两个
  16位宽的match信号表示查找结果。\par
  \begin{lstlisting}[language=verilog]
    genvar i;
generate
    for (i = 0; i < TLBNUM; i = i + 1) begin
        assign match0[i] = (s0_vppn[18:9]==tlb_vppn[i][18:9]) && (tlb_ps4MB[i] || s0_vppn[8:0]==tlb_vppn[i][8:0]) && ((s0_asid==tlb_asid[i]) || tlb_g[i]);
        assign match1[i] = (s1_vppn[18:9]==tlb_vppn[i][18:9]) && (tlb_ps4MB[i] || s1_vppn[8:0]==tlb_vppn[i][8:0]) && ((s1_asid==tlb_asid[i]) || tlb_g[i]);
    end
endgenerate
  \end{lstlisting}
  匹配的逻辑是:对于每一个页表项,一个是要查看比较部分的全局标志位G和地址空间表示位
  ASID;除此之外,还需要查看比较部分的页大小PS位和虚双页号VPPN位,如
  果PS位为1,表示页大小为4MB(两个物理页大小均为2MB),此时只需要比
  较VPPN的高10位是否与s_vppn的高10位相等,若相等则查找成功;如果PS
  位为0,表示页大小为4KB(两个物理页大小均为4KB),此时需要比较VPPN
  和s_vppn的全部位数。如果以上全部满足则查找成功。\par

  然后得到found,portindex,port,ps信号(端口0为例),其中含义如下:\par
  found:是否匹配到。\par
  index:匹配到的页表项的项数。\par
  port:双页中的哪一页,页大小为2MB时通过vppn的第8位来判断,4KB时通过输入的va_bit12来判断。\par
  ps:页大小,2MB为21,4KB为12。\par
  \begin{lstlisting}[language=verilog]
assign s0_index = match0[0]  ? 4'b0000 :...    
assign s1_index = match1[0]  ? 4'b0000 :...
                 
assign s0_found = |match0[TLBNUM-1:0];
assign s1_found = |match1[TLBNUM-1:0];

assign s0_ps = tlb_ps4MB[s0_index] ? 6'b010101 : 6'b001100;// 21 or 12 
assign s1_ps = tlb_ps4MB[s1_index] ? 6'b010101 : 6'b001100;

assign s0_port = tlb_ps4MB[s0_index] ? s0_vppn[8] : s0_va_bit12;
assign s1_port = tlb_ps4MB[s1_index] ? s1_vppn[8] : s1_va_bit12;

  \end{lstlisting}
  最后通过index和port从页表中得到物理转换信息并输出:
  \begin{lstlisting}[language=verilog]
    assign s0_ppn = s0_port ? tlb_ppn1[s0_index] : tlb_ppn0[s0_index];
    assign s1_ppn = s1_port ? tlb_ppn1[s1_index] : tlb_ppn0[s1_index];
    assign s0_plv = s0_port ? tlb_plv1[s0_index] : tlb_plv0[s0_index];
    assign s1_plv = s1_port ? tlb_plv1[s1_index] : tlb_plv0[s1_index];
    assign s0_mat = s0_port ? tlb_mat1[s0_index] : tlb_mat0[s0_index];
    assign s1_mat = s1_port ? tlb_mat1[s1_index] : tlb_mat0[s1_index];
    assign s0_d = s0_port ? tlb_d1[s0_index] : tlb_d0[s0_index];
    assign s1_d = s1_port ? tlb_d1[s1_index] : tlb_d0[s1_index];
    assign s0_v = s0_port ? tlb_v1[s0_index] : tlb_v0[s0_index];
    assign s1_v = s1_port ? tlb_v1[s1_index] : tlb_v0[s1_index];
      \end{lstlisting}
\end{enumerate}



\noindent
$\bullet$
\textbf{添加TLB相关指令和CSR寄存器}。
\vspace{1ex}
\begin{enumerate}
  \item CSR寄存器的添加

  按照书上的指示添加 TLBIDX、TLBEHI、TLBELO0、TLBELO1、ASID、TLBRENTRY CSR 寄存器即可。\par
  \begin{lstlisting}[language=verilog]
/*------------------------------TLBIDX-------------------------------------------------*/
    always @(posedge clk) begin
        if (~resetn) 
            csr_tlbidx_index <= {($clog2(TLBNUM)) {1'b0}};
        else if (tlbsrch_en & tlbsrch_found) 
            csr_tlbidx_index <= tlbsrch_idx;
        else if(csr_we && csr_num == `CSR_TLBIDX)
            csr_tlbidx_index <=   csr_wmask[`CSR_TLBIDX_INDEX]  & csr_wvalue[`CSR_TLBIDX_INDEX]
                                | ~csr_wmask[`CSR_TLBIDX_INDEX]  & csr_tlbidx_index;       
    end

    always @(posedge clk)begin
        if(~resetn)
            csr_tlbidx_ps <= 6'b0;
        else if(tlbrd_en)
            csr_tlbidx_ps <= {6{tlbrd_valid}} & tlbrd_ps;
        else if(csr_we && csr_num == `CSR_TLBIDX)
            csr_tlbidx_ps <=     csr_wmask[`CSR_TLBIDX_PS]  & csr_wvalue[`CSR_TLBIDX_PS]
                                | ~csr_wmask[`CSR_TLBIDX_PS]  & csr_tlbidx_ps;  
    end

    always @(posedge clk) begin
        if(~resetn)
            csr_tlbidx_ne <= 1'b0;
        else if(tlbsrch_en)    // not found
            csr_tlbidx_ne <= ~tlbsrch_found;
        else if(tlbrd_en)
            csr_tlbidx_ne <= ~tlbrd_valid;
        else if(csr_we && csr_num == `CSR_TLBIDX)
            csr_tlbidx_ne <=     csr_wmask[`CSR_TLBIDX_NE]  & csr_wvalue[`CSR_TLBIDX_NE]
                                | ~csr_wmask[`CSR_TLBIDX_NE]  & csr_tlbidx_ne;  
    end

    assign tlbrd_idx = csr_tlbidx_index;

/*--------------------------------TLBEHI------------------------------------------------------------*/
    always @(posedge clk)begin
        if(~resetn)
            csr_tlbehi_vppn <= 19'b0;
        else if(tlbrd_en)
            csr_tlbehi_vppn <= {19{tlbrd_valid}} & tlbrd_vppn;
        else if( wb_ex & (wb_ecode == `ECODE_TLBR || wb_ecode == `ECODE_PIL|| wb_ecode == `ECODE_PIS 
                    || wb_ecode == `ECDOE_PIF || wb_ecode == `ECODE_PME || wb_ecode == `ECODE_PPI))
            csr_tlbehi_vppn <= wb_badv[31:13];

        else if(csr_we && csr_num == `CSR_TLBEHI)
            csr_tlbehi_vppn <=     csr_wmask[`CSR_TLBEHI_VPPN]  & csr_wvalue[`CSR_TLBEHI_VPPN]
                                | ~csr_wmask[`CSR_TLBEHI_VPPN]  & csr_tlbehi_vppn;  
    end

    assign wire_csr_tlbehi_vppn = csr_tlbehi_vppn;
/*--------------------------------TLBELO--------------------------------------------------------*/
    always @(posedge clk)begin
        if(~resetn)begin
            {csr_tlbelo0_v,csr_tlbelo0_d,csr_tlbelo0_plv,csr_tlbelo0_mat,csr_tlbelo0_g,csr_tlbelo0_ppn} 
                <= 27'b0;
        end
        else if(tlbrd_en)begin
            {csr_tlbelo0_v,csr_tlbelo0_d,csr_tlbelo0_plv,csr_tlbelo0_mat,csr_tlbelo0_g,csr_tlbelo0_ppn} 
                <= {27{tlbrd_valid}} & {tlbrd_v0,tlbrd_d0,tlbrd_plv0,tlbrd_mat0,tlbrd_g,tlbrd_ppn0}; 
        end
        else if(csr_we && csr_num == `CSR_TLBELO0) begin
            csr_tlbelo0_v <= csr_wmask[`CSR_TLBELO_V] & csr_wvalue[`CSR_TLBELO_V]
                            | ~csr_wmask[`CSR_TLBELO_V] & csr_tlbelo0_v;
            csr_tlbelo0_d <= csr_wmask[`CSR_TLBELO_D] & csr_wvalue[`CSR_TLBELO_D]
                             | ~csr_wmask[`CSR_TLBELO_D] & csr_tlbelo0_d;
            csr_tlbelo0_plv <= csr_wmask[`CSR_TLBELO_PLV] & csr_wvalue[`CSR_TLBELO_PLV]
                            | ~csr_wmask[`CSR_TLBELO_PLV] & csr_tlbelo0_plv;
            csr_tlbelo0_mat <= csr_wmask[`CSR_TLBELO_MAT] & csr_wvalue[`CSR_TLBELO_MAT]
                            | ~csr_wmask[`CSR_TLBELO_MAT] & csr_tlbelo0_mat;
            csr_tlbelo0_g <=    csr_wmask[`CSR_TLBELO_G] & csr_wvalue[`CSR_TLBELO_G]
                            | ~csr_wmask[`CSR_TLBELO_G] & csr_tlbelo0_g;
            csr_tlbelo0_ppn <= csr_wmask[`CSR_TLBELO_PPN] & csr_wvalue[`CSR_TLBELO_PPN]
                            | ~csr_wmask[`CSR_TLBELO_PPN] & csr_tlbelo0_ppn;
        end
    end

    always @(posedge clk) begin
        if(~resetn)begin
            {csr_tlbelo1_v,csr_tlbelo1_d,csr_tlbelo1_plv,csr_tlbelo1_mat,csr_tlbelo1_g,csr_tlbelo1_ppn} 
                <= 27'b0;
        end
        else if(tlbrd_en)begin
            {csr_tlbelo1_v,csr_tlbelo1_d,csr_tlbelo1_plv,csr_tlbelo1_mat,csr_tlbelo1_g,csr_tlbelo1_ppn} 
                <= {27{tlbrd_valid}} & {tlbrd_v1,tlbrd_d1,tlbrd_plv1,tlbrd_mat1,tlbrd_g,tlbrd_ppn1}; 
        end
        else if(csr_we && csr_num == `CSR_TLBELO1) begin
            csr_tlbelo1_v <= csr_wmask[`CSR_TLBELO_V] & csr_wvalue[`CSR_TLBELO_V]
                            | ~csr_wmask[`CSR_TLBELO_V] & csr_tlbelo1_v;
            csr_tlbelo1_d <= csr_wmask[`CSR_TLBELO_D] & csr_wvalue[`CSR_TLBELO_D]
                            | ~csr_wmask[`CSR_TLBELO_D] & csr_tlbelo1_d;
            csr_tlbelo1_plv <= csr_wmask[`CSR_TLBELO_PLV] & csr_wvalue[`CSR_TLBELO_PLV]
                            | ~csr_wmask[`CSR_TLBELO_PLV] & csr_tlbelo1_plv;
            csr_tlbelo1_mat <= csr_wmask[`CSR_TLBELO_MAT] & csr_wvalue[`CSR_TLBELO_MAT]
                            | ~csr_wmask[`CSR_TLBELO_MAT] & csr_tlbelo1_mat;
            csr_tlbelo1_g <=    csr_wmask[`CSR_TLBELO_G] & csr_wvalue[`CSR_TLBELO_G]
                            | ~csr_wmask[`CSR_TLBELO_G] & csr_tlbelo1_g;
            csr_tlbelo1_ppn <= csr_wmask[`CSR_TLBELO_PPN] & csr_wvalue[`CSR_TLBELO_PPN]
                            | ~csr_wmask[`CSR_TLBELO_PPN] & csr_tlbelo1_ppn;
        end
    end
/*-------------------------------------ASI0-----------------------------------------------*/
    always @(posedge clk) begin
        if(~resetn)
            csr_asid_asid <= 10'b0;
        else if(tlbrd_en)
            csr_asid_asid <= {10{tlbrd_valid}} & tlbrd_asid;
        else if(csr_we && csr_num == `CSR_ASID)
            csr_asid_asid <=     csr_wmask[`CSR_ASID_ASID]  & csr_wvalue[`CSR_ASID_ASID]
                                | ~csr_wmask[`CSR_ASID_ASID]  & csr_asid_asid;  
    end

    assign csr_asid_asidbits = 6'h0a;   // 10 bit asid
    assign wire_csr_asid =csr_asid_asid;
/*----------------------------------TLBRENTRY-----------------------------------------------*/
    always @(posedge clk)begin
        if(~resetn)
            csr_tlbrentry_pa <= 26'b0;
        else if(csr_we && csr_num == `CSR_TLBRENTRY)
            csr_tlbrentry_pa <=     csr_wmask[`CSR_TLBRENTRY_PA]  & csr_wvalue[`CSR_TLBRENTRY_PA]
                                | ~csr_wmask[`CSR_TLBRENTRY_PA]  & csr_tlbrentry_pa; 
    end

    // tlbwr/tlbfill output
    assign csr_tlb_ne = (csr_estat_ecode == 6'h3F) ? 0 : csr_tlbidx_ne;
    assign csr_tlb_index = csr_tlbidx_index;
    assign csr_tlb_asid = csr_asid_asid;
    assign csr_tlb_g = csr_tlbelo0_g & csr_tlbelo1_g; // only if the G bit in both TLBELO0 and TLBELO1 is 1
    assign csr_tlb_ps = csr_tlbidx_ps;
    assign csr_tlb_vppn = csr_tlbehi_vppn;
    assign {csr_tlb_v0, csr_tlb_d0, csr_tlb_mat0, csr_tlb_plv0, csr_tlb_ppn0} = 
        {csr_tlbelo0_v,csr_tlbelo0_d,csr_tlbelo0_mat,csr_tlbelo0_plv,csr_tlbelo0_ppn};
    assign {csr_tlb_v1, csr_tlb_d1, csr_tlb_mat1, csr_tlb_plv1, csr_tlb_ppn1} = 
        {csr_tlbelo1_v,csr_tlbelo1_d,csr_tlbelo1_mat,csr_tlbelo1_plv,csr_tlbelo1_ppn};
  \end{lstlisting}
  
  \item TLB相关指令的实现

  需要处理每个指令在对应流水级发出控制信号给TLB模块。\par
  与TLB相关的指令类型在ID级用id\_tlb\_op的每一位来记录。\par
  \begin{lstlisting}[language=verilog]
    assign id_tlb_op = {inst_tlbsrch,inst_tlbwr,inst_tlbfill,inst_tlbrd,inst_invtlb};
  \end{lstlisting}
  如果TLBSRCH指令的执行能够复用其中一套查找逻辑就可以避免新增大量查找逻辑,同时希望能确保复用的时候不阻塞其他指令访问TLB模块;INVTLB 指令的源操作数都来自通用寄存器或立即数,并且它需要(部分)复用 TLB 模块中已有 的查找逻辑。所以在EX阶段发送TLBSRCH和INVTLB指令请求。而TLBWR、TLBFILL、TLBRD则考虑到数据冲突的问题选择在WB阶段发送指令请求。\par
  然后根据这些发送的指令请求以及CSRFILE文件中传出来的部分CSR寄存器的值,来对TLB模块进行接口连接。\par
  \begin{lstlisting}[language=verilog]
        tlb u_tlb(
        .clk                (aclk),

        .s0_vppn            (if_s0_vppn),
        .s0_va_bit12        (if_s0_va_bit12),
        .s0_asid            (csr_tlb_asid),
        .s0_found           (s0_found),
        .s0_index           (s0_index),
        .s0_ppn             (s0_ppn),
        .s0_ps              (s0_ps),
        .s0_plv             (s0_plv),
        .s0_mat             (s0_mat),
        .s0_d               (s0_d),
        .s0_v               (s0_v),

        .s1_vppn            (ex_s1_vppn),
        .s1_va_bit12        (ex_s1_va_bit12),
        .s1_asid            (ex_s1_asid),
        .s1_found           (s1_found),
        .s1_index           (s1_index),
        .s1_ppn             (s1_ppn),
        .s1_ps              (s1_ps),
        .s1_plv             (s1_plv),
        .s1_mat             (s1_mat),
        .s1_d               (s1_d),
        .s1_v               (s1_v),

        .invtlb_valid       (ex_invtlb_valid),
        .invtlb_op          (ex_invtlb_op),

        .we                 (wb_tlbwr_en || wb_tlbfill_en),
        .w_index            (csr_tlb_index),
        .w_e                (~csr_tlb_ne),
        .w_vppn             (csr_tlb_vppn),
        .w_ps               (csr_tlb_ps),
        .w_asid             (csr_tlb_asid),
        .w_g                (csr_tlb_g),
        .w_ppn0             (csr_tlb_ppn0),
        .w_plv0             (csr_tlb_plv0),
        .w_mat0             (csr_tlb_mat0),
        .w_d0               (csr_tlb_d0),
        .w_v0               (csr_tlb_v0),
        .w_ppn1             (csr_tlb_ppn1),
        .w_plv1             (csr_tlb_plv1),
        .w_mat1             (csr_tlb_mat1),
        .w_d1               (csr_tlb_d1),
        .w_v1               (csr_tlb_v1),

        .r_index            (csr_tlb_index),
        .r_e                (r_e),
        .r_vppn             (r_vppn),
        .r_ps               (r_ps),
        .r_asid             (r_asid),
        .r_g                (r_g),
        .r_ppn0             (r_ppn0),
        .r_plv0             (r_plv0),
        .r_mat0             (r_mat0),
        .r_d0               (r_d0),
        .r_v0               (r_v0),
        .r_ppn1             (r_ppn1),
        .r_plv1             (r_plv1),
        .r_mat1             (r_mat1),
        .r_d1               (r_d1),
        .r_v1               (r_v1)
    );
  \end{lstlisting}
  另外,这些新增指令还需要进行冲突处理。\par
  对于TLBSRCH引起的冲突,使用阻塞的方法来进行处理。
  \begin{lstlisting}[language=verilog]
    assign ex_ready_go      = ~block & alu_complete & (~data_sram_req | data_sram_req & data_sram_addr_ok);//等待alu完成运算
    assign block            =( ex_tlb_op[4] & mem_srch_conflict) |(ex_tlb_op[4] & wb_srch_conflict);
  \end{lstlisting}
  其他指令引起的冲突,则使用重取方式做处理。其中wb\_refetch\_flush信号用于刷新流水线,mem\_refetch记录是否需要重取。
  \begin{lstlisting}[language=verilog]
    assign mem_refetch =    mem_tlb_op[3]   // inst_tlbwr
                            || mem_tlb_op[2]    // inst_tlbfill
                            || mem_tlb_op[1]    // inst_tlbrd
                            || mem_tlb_op[0]   // inst_invtlb
                            || (mem_csr_we && (mem_csr_num ==14'h18 // ASID
                                            || mem_csr_num ==14'h0 // CRMD
                                            || mem_csr_num ==14'h180// DMW0
                                            ||mem_csr_num ==14'h181))// DMW1
                            ;
    assign wb_refetch_flush =    wb_valid && 
                                (wb_tlb_op[3]   // inst_tlbwr
                                || wb_tlb_op[2]    // inst_tlbfill
                                || wb_tlb_op[1]    // inst_tlbrd
                                || wb_tlb_op[0]   // inst_invtlb
                                || (wb_csr_we && (wb_csr_num ==14'h18 // ASID
                                            || wb_csr_num ==14'h0 // CRMD
                                            || wb_csr_num ==14'h180// DMW0
                                            ||wb_csr_num ==14'h181)));// DMW1
  \end{lstlisting}

\end{enumerate}


\noindent
$\bullet$
\textbf{添加TLB相关例外支持,添加虚实地址映射功能}。
\vspace{1ex}
\begin{enumerate}
  \item 为cpu增加虚实映射功能
  
  在if和ex级发出访存请求的虚拟地址,同时送往直接地址翻译,直接窗口映射和tlb地址映射翻译三处,同时进行各自的
  虚实地址转换,然后再根据CSR寄存器中的值进行选择。

  以if级发出指令访存信号为例,通过组合逻辑,在同一个clk内生成三种地址翻译的结果,最后通过多路选择器进行选择。

  访存物理地址pre\_pc\_pa首先根据控制寄存器中的 crmd.pg中的值选择当前为直接地址地址翻译模式(物理地址 = 虚拟地址 = pre\_pc)
  ,还是映射地址翻译模式(物理地址 = pre\_pc\_map)。

  映射地址翻译(pre\_pc\_map)首先判断是否命中直接映射窗口,若不是,则使用tlb地址映射结构。
  
  第一行将虚拟地址 pre\_pc [31:12] 发送到tlb进行tlb地址翻译,tlb中的组合逻辑查找结果s0\_ps, s0\_ppn, s0\_found
  
  \begin{lstlisting}[language=verilog]


    assign {s0_vppn, s0_va_bit12} = pre_pc[31:12];// output to tlb
                            
    assign hit_dmw0 =           csr_dmw0_plv_met & csr_dmw0_vseg == pre_pc[31:29];
    assign hit_dmw1 =           csr_dmw1_plv_met & csr_dmw1_vseg == pre_pc[31:29];

    assign pre_pc_pa =          csr_crmd_pg ? pre_pc_map    // enable mapping
    :pre_pc;                    // direct translate

    assign pre_pc_map =  hit_dmw0 ? {csr_dmw0_pseg, pre_pc[28:0]}         // dierct map windows 0
                         :hit_dmw1? {csr_dmw1_pseg, pre_pc[28:0]}         // direct map windows 1
                         :(s0_ps == 6'b010101) ? {s0_ppn[19:9], pre_pc[20:0]}   // tlb map: ps 4Mb
                         :{s0_ppn,pre_pc[11:0]};                             // tlb map : ps 4kb
  \end{lstlisting}
  
  \item 为cpu添加tlb相关例外支持
  
  在每个流水级中设置excep\_en , excep\_ecode, excep\_esubdcode分别表示当前流水级是否触发异常,和异常的种类。

  以在ex流水级中为例:
\begin{lstlisting}[language=verilog]
  assign ex_excep_en =        ex_excep_ALE | ex_excep_TLBR|ex_excep_PIL| ex_excep_PIS| ex_excep_PPI|ex_excep_PME| id_excep_en;

  assign ex_excep_TLBR =       //TLB 重填例外
      (ex_res_from_mem | ex_mem_we) &csr_crmd_pg & ~hit_dmw0 & ~hit_dmw1 & ~s1_found;   
  assign ex_excep_PIL =       // load 操作页无效例外
      (ex_res_from_mem) &csr_crmd_pg & ~hit_dmw0 & ~hit_dmw1 & s1_found & ~s1_v;  
  assign ex_excep_PIS =       // store 操作页无效例外
      (ex_mem_we) &csr_crmd_pg & ~hit_dmw0 & ~hit_dmw1 & s1_found & ~s1_v; 
  assign ex_excep_PPI =       // 页特权等级不合规例外
      (ex_res_from_mem | ex_mem_we) & csr_crmd_pg & ~hit_dmw0 & ~hit_dmw1 & s1_found & s1_v & (s1_plv < csr_crmd_plv);
  assign ex_excep_PME =        //页修改例外
      (ex_res_from_mem | ex_mem_we) & csr_crmd_pg & ~hit_dmw0 & ~hit_dmw1 & s1_found & s1_v & (s1_plv >= csr_crmd_plv) & ~s1_d;
  
  assign ex_badv =            (id_excep_en) ? id_badv
                              : ex_alu_result;
  assign ex_esubcode =        (id_excep_en) ? id_esubcode
                              :9'b0;
  assign ex_ecode =           (id_excep_en) ? id_ecode
                              :ex_excep_ALE ? 6'h9
                              :ex_excep_TLBR ?6'h3f     // tlb refill
                              :ex_excep_PIL ?6'h1
                              :ex_excep_PIS ?6'h2
                              :ex_excep_PPI ?6'h7
                              :6'h4;  // pme
\end{lstlisting}
 所有tlb相关的异常,触发的必要条件都有:当前指令是访存指令(load/store), 当前虚拟地址翻译模式不是直接
  地址翻译模式,且不命中直接映射窗口。


  即 \verb| (ex_res_from_mem| |\verb| ex_mem_we) &csr_crmd_pg & ~hit_dmw0 & ~hit_dmw1|
  \begin{enumerate}
    \item TLB 重填例外 
    
    在tlb没有找到对应的页表项,即\verb|s1_found == 0|

    \item load 操作页无效例外, store 操作页无效例外
    
    在tlb中找到对应的页表项,但是页表项的有效位为 0, 即\verb |s1_found & ~s1_v|
    
    \item  页特权等级不合
    
    在tlb中找到对应的页表项,页表项的有效,但当前特权等级为plv3, 但是页表项所需要特权等级为plv0, 
    即 \verb|s1_found & s1_v & (s1_plv < csr_crmd_plv)|

    \item 页修改例外
    在tlb中找到对应的页表项,页表项的有效,且特权等级满足要求, 但是页表项脏位为0, 
    即 \verb|s1_found & s1_v & (s1_plv >= csr_crmd_plv) & ~s1_d|
  \end{enumerate}
  由上面的分析可知,与tlb相关的异常触发条件互斥,不用考虑不同异常间的优先级的问题。
\end{enumerate}


\section{实验过程中遇到的问题、对问题的思考过程及解决方法(比如RTL代码中出现的逻辑bug,逻辑仿真和FPGA调试过程中的难点等)}

\noindent
$\bullet$
\textbf{TLB 搜索match出错}。

TLB测试的过程中发现了一个错误:在TLB的search测试的第一个节点就卡住了,经过仔细排查发现是pagesize出错了,
后来经仔细看测试数据发现pagesize大页是2MB(大小是21),这与讲义上的要求4MB不同,我百思不得其解,后来仔细查看指令手册发现了
一行小字:
\begin{figure}[H]
    \centering
    \includegraphics[width=13cm]{fig/3.png}
  \end{figure}
这下恍然大悟,将22改为21之后,然而,测试依然出错。
仔细追根溯源,发现s0_ppn匹配出错的原因是s0_port的实现出错力,理应是根据s0_vppn[8](有效的最后一位)
这一位来判断大页的时候选择的是双页中的哪一页,然而这里写成了s0_vppn[9]...将9改为8之后,测试通过。
\begin{lstlisting}[language=verilog]
assign s0_port = tlb_ps4MB[s0_index] ? s0_vppn[9] : s0_va_bit12;
assign s1_port = tlb_ps4MB[s1_index] ? s1_vppn[9] : s1_va_bit12;

assign s0_ppn = s0_port ? tlb_ppn1[s0_index] : tlb_ppn0[s0_index];
assign s1_ppn = s1_port ? tlb_ppn1[s1_index] : tlb_ppn0[s1_index];
\end{lstlisting}
之所以这里会出错,是因为电子版的讲义示例代码出现了错误,如下:
\begin{figure}[H]
    \centering
    \includegraphics[width=15cm]{fig/4.png}
  \end{figure}
这里的匹配就让人误以为是s0_vppn[9]决定了大页的时候选择的是双页中的哪一页,当然,这里的代码也出错了
具体的修改如下,修改后通过测试。
\begin{lstlisting}[language=verilog]
    generate
    for (i = 0; i < TLBNUM; i = i + 1) begin
        assign match0[i] = (s0_vppn[18:9]==tlb_vppn[i][18:9]) && (tlb_ps4MB[i] || s0_vppn[8:0]==tlb_vppn[i][8:0]) && ((s0_asid==tlb_asid[i]) || tlb_g[i]);
        assign match1[i] = (s1_vppn[18:9]==tlb_vppn[i][18:9]) && (tlb_ps4MB[i] || s1_vppn[8:0]==tlb_vppn[i][8:0]) && ((s1_asid==tlb_asid[i]) || tlb_g[i]);
    end
endgenerate
\end{lstlisting}

\noindent
$\bullet$
\textbf{if级错误丢弃异常处理函数的第一条指令}。

要求在发生mmu相关异常的时候,不能向总线发出请求,所以在pre\_if级中识别到pc的虚拟地址发生异常,则将 inst\_sram\_req
拉低。

但是导致了异常传到wb级时,发出更新流水线信号flush时,将if级误判为还在等待指令返回(实际上pre\_if级没有发出请求),则
将inst_cancel信号(表示要丢弃接收到的第一条指令)拉高,导致丢弃了异常处理含数的第一条指令。
\begin{figure}[H]
  \centering
  \includegraphics[width=15cm]{fig/fig1.png}
  \caption{错误拉高inst\_cancel}
\end{figure}
修改:在inst\_cancel拉高的逻辑中,判断if级是否有异常,若有,则说明if级没有已经发出请求,但是没有接受到的总线事务,
不用拉高inst\_cancel。
\begin{lstlisting}[language=verilog]
  /* 清空流水线时,第一个指令需要丢弃*/
  always @(posedge clk) begin
      if(~resetn)
          inst_cancel <= 1'b0;
      else if (   (if_valid & ~if_ir_valid & ~inst_sram_data_ok & ~if_excep_en  // if正在等待指令返回, 加入if级是否有异常的判断
                  |pre_if_reqed_reg & ~inst_sram_data_ok) // pre_if 级发出请求,但是数据没有返回,也还没有进入if级
              & (flush | br_taken))
          inst_cancel <= 1'b1;
      else if(inst_sram_data_ok)      // 异常后第一个需要被舍弃的指令返回
          inst_cancel <= 1'b0;
  end
\end{lstlisting}


      
\vspace{1ex}

\section{小组成员分工合作情况}
王敬华负责TLB模块实现,部分TLB指令的添加和冲突的处理。

李霄宇负责TLB相关指令的添加,部分冲突的处理,接口连接。

艾华春负责添加TLB相关寄存器和TLB相关的例外支持,为cpu增加虚实映射功能。

实验报告为根据每人负责代码的部分,写相应部分的报告。
\end{document}